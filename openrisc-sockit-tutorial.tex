%%%%%%%%%%%%%%%%%%%%%%%%%%%%%%%%%%%%%%%%%
% Journal Article
% LaTeX Template
% Version 1.3 (9/9/13)
%
% This template has been downloaded from:
% http://www.LaTeXTemplates.com
%
% Original author:
% Frits Wenneker (http://www.howtotex.com)
%
% License:
% CC BY-NC-SA 3.0 (http://creativecommons.org/licenses/by-nc-sa/3.0/)
%
%%%%%%%%%%%%%%%%%%%%%%%%%%%%%%%%%%%%%%%%%

%----------------------------------------------------------------------------------------
%	PACKAGES AND OTHER DOCUMENT CONFIGURATIONS
%----------------------------------------------------------------------------------------

\documentclass[twoside]{article}

\usepackage{lipsum} % Package to generate dummy text throughout this template

\usepackage[sc]{mathpazo} % Use the Palatino font
\usepackage[T1]{fontenc} % Use 8-bit encoding that has 256 glyphs
\linespread{1.05} % Line spacing - Palatino needs more space between lines
\usepackage{microtype} % Slightly tweak font spacing for aesthetics
\usepackage{alltt}

\usepackage[hmarginratio=1:1,top=32mm,columnsep=20pt]{geometry} % Document margins
\usepackage{multicol} % Used for the two-column layout of the document
\usepackage[hang, small,labelfont=bf,up,textfont=it,up]{caption} % Custom captions under/above floats in tables or figures
\usepackage{booktabs} % Horizontal rules in tables
\usepackage{float} % Required for tables and figures in the multi-column environment - they need to be placed in specific locations with the [H] (e.g. \begin{table}[H])
\usepackage{hyperref} % For hyperlinks in the PDF

\usepackage{lettrine} % The lettrine is the first enlarged letter at the beginning of the text
\usepackage{paralist} % Used for the compactitem environment which makes bullet points with less space between them

\usepackage{abstract} % Allows abstract customization
\renewcommand{\abstractnamefont}{\normalfont\bfseries} % Set the "Abstract" text to bold
\renewcommand{\abstracttextfont}{\normalfont\small\itshape} % Set the abstract itself to small italic text

\usepackage{titlesec} % Allows customization of titles
\renewcommand\thesection{\Roman{section}} % Roman numerals for the sections
\renewcommand\thesubsection{\Roman{subsection}} % Roman numerals for subsections
\titleformat{\section}[block]{\large\scshape\centering}{\thesection.}{1em}{} % Change the look of the section titles
\titleformat{\subsection}[block]{\large}{\thesubsection.}{1em}{} % Change the look of the section titles

\usepackage{fancyhdr} % Headers and footers
\pagestyle{fancy} % All pages have headers and footers
\fancyhead{} % Blank out the default header
\fancyfoot{} % Blank out the default footer
\fancyhead[C]{Getting started with OpenRISC on the SocKit} % Custom header text
\fancyfoot[RO,LE]{\thepage} % Custom footer text

%----------------------------------------------------------------------------------------
%	TITLE SECTION
%----------------------------------------------------------------------------------------

\title{\vspace{-15mm}\fontsize{24pt}{10pt}\selectfont\textbf{Getting
    started with OpenRISC on the SocKit}} % Article title

\author{
\large
\textsc{Hans Baier}\thanks{Thanks to Stephan Kristiansson for his
  generous help and Kevin Mehall for his excellent tutorial on the de0\_nano}\\[2mm] 
\normalsize \href{mailto:hansfbaier@gmail.com}{hansfbaier@gmail.com} % Your email address
\vspace{-5mm}
}
\date{}

%----------------------------------------------------------------------------------------

\begin{document}

\maketitle % Insert title

\thispagestyle{fancy} % All pages have headers and footers

%----------------------------------------------------------------------------------------
%	ABSTRACT
%----------------------------------------------------------------------------------------

\begin{abstract}

\noindent Getting started with OpenRISC can be quite daunting for
beginners. There is so much to learn. Fortunately the community is
very helpful. In order to diminish duplication of effort, this
tutorial was conceived out of the generous hours of support that
Stefan provided over IRC. This tutorial also provides some of the
supplemental files, for convenience.

\end{abstract}

%----------------------------------------------------------------------------------------
%	ARTICLE CONTENTS
%----------------------------------------------------------------------------------------

%\begin{multicols}{2} % Two-column layout throughout the main article text

\section{Introduction}

\lettrine[nindent=0em,lines=3]{O} penRISC is an amazing learning
plattform: You can run Open Source Software on Open Source Hardware
and see down to the metal, how hardware is designed, drivers are
written, and how software and hardware interact.  

%------------------------------------------------

\section{Prerequisites}
This tutorial assumes you are using a GNU/Linux Distribution. This
tutorial is based on Ubuntu 12.04 (precise).

\begin{compactitem}
\item Experience with Linux systems
\item Compiling software from source
\item Basic understanding of digital logic, communication protocols
  and electronics.
\item Familiarity with Verilog and C
\end{compactitem}

You will need a Linux workstation with a 64 Bit operating system, 8GB
of RAM and about 20-30GB free hard drive space. You will also need
terasic's
\href{http://www.terasic.com.tw/cgi-bin/page/archive.pl?Language=English\&No=816}{SocKit
  Development Board}.

%------------------------------------------------

\section{Step By Step}
\begin{enumerate}
\item Install \href{http://dl.altera.com/?edition=web}{Altera Quartus
    II Web Edition} at least version 13.1. For this tutorial we assume
  you installed it in \texttt{/opt/altera/}

\item Download the
  \href{http://www.terasic.com.tw/cgi-bin/page/archive.pl?Language=English\&CategoryNo=165\&No=816\&PartNo=4}{resources}
  for the development board. Find out what board revision you have
  (first PDF on the resources page), and download the SocKit System CD
  for the correct board revision. This will be most likely be Revision
  C at the time of this writing:
  \begin{alltt}
\$ wget -c http://www.terasic.com/downloads/cd-rom/sockit/SoCKit\_V.1.0.0\_System.zip
  \end{alltt}

\item Install development tools:
  \begin{alltt}
\$ sudo apt-get install build-essential libmpc-dev libgmp3-dev libmpfr-dev  \textbackslash
                       lzop libsdl1.2-dev xterm automake libtool git-core git
  \end{alltt}

\item Create the directory where we put openrisc related files:
  \begin{alltt}
\$ mkdir ~/openrisc
  \end{alltt}

\item Create a file with necessary environment variables:
  \begin{alltt}
\$ cd ~/openrisc
\$ editor altera_env.sh
\hrulefill
export ALTERA_PATH=/opt/altera
export PATH=$PATH:$ALTERA_PATH/quartus/bin
\hrulefill
\$ source altera_env.sh
  \end{alltt}
  You may want to source this file every time you want to use Altera's
  tools.

\item Get \texttt{orpsoc} which is the OpenRISC-SoC-Builder and
  \texttt{orpsoc-cores}, which contains the IP-cores and board
  configurations:
  \begin{alltt}
\$ git clone https://github.com/openrisc/orpsoc.git
\$ git clone https://github.com/openrisc/orpsoc-cores.git
  \end{alltt}

\end{enumerate}


%----------------------------------------------------------------------------------------
%	REFERENCE LIST
%----------------------------------------------------------------------------------------

\begin{thebibliography}{99} % Bibliography - this is intentionally simple in this template

\bibitem[Figueredo and Wolf, 2009]{Figueredo:2009dg}
Figueredo, A.~J. and Wolf, P. S.~A. (2009).
\newblock Assortative pairing and life history strategy - a cross-cultural
  study.
\newblock {\em Human Nature}, 20:317--330.
 
\end{thebibliography}

%----------------------------------------------------------------------------------------

%\end{multicols}

\end{document}